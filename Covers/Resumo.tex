%----------------------------------------------------------------------------------------
%   Resumo (Portuguese Abstract)
%----------------------------------------------------------------------------------------

\selectlanguage{portuguese}
\pdfbookmark[1]{Resumo}{Resumo}
\chapter*{Resumo}

% Add your Portuguese abstract content here
Este trabalho apresenta um estudo sobre a estimação da condição de baterias em aplicações automotivas e ferroviárias utilizando inteligência artificial. O projeto foca na monitorização da saúde de baterias através de parâmetros como Estado de Carga (SoC), Estado de Saúde (SoH) e Vida Útil Remanescente (RUL). Utilizando técnicas de aprendizagem automática e análise de séries temporais, desenvolvemos modelos preditivos capazes de identificar padrões de degradação e prever falhas em sistemas de baterias. Os resultados demonstram a eficácia das abordagens baseadas em IA para melhorar a confiabilidade e segurança dos sistemas de energia em aplicações críticas.

\vfill

\noindent
\textbf{Palavras-chave:} Inteligência Artificial, Baterias, Estado de Saúde, Monitorização, Aprendizagem Automática

\selectlanguage{English}
