%*******************************************************
% Abstract
%*******************************************************


\renewcommand{\abstractname}{Resumo}
\markboth{\spacedlowsmallcaps{\abstractname}}{\spacedlowsmallcaps{\abstractname}}
\refstepcounter{dummy}
\addcontentsline{toc}{chapter}{\abstractname}


\begingroup
\let\clearpage\relax
\let\cleardoublepage\relax

\chapter*{Resumo}
Este relatório apresenta uma panorâmica e uma análise dos métodos baseados na inteligência artificial para monitorizar o estado de saude de baterias nas indústrias automóvel e ferroviária. A análise centra-se na estimativa do estado de saúde da bateria (SOH), do estado de carga (SOC) e da vida útil restante (RUL) utilizando arquitecturas avançadas de redes neuronais. Descreve todo o processo de previsão do estado de saúde da bateria, desde o pré processamento de dados até à implementação dos modelos de aprendizagem profunda para a estimativa de vários parâmetros das baterias. São analisadas várias abordagens metodológicas, desde os modelos tradicionais baseados na física até às arquitecturas de análise de séries temporais mais avançadas.

É efectuada uma avaliação de diferentes arquitecturas de redes neurais, incluindo uma análise comparativa das redes Transformer e Mixture of Experts (MoE). Com base nestas conclusões, a arquitetura TimesNet foi escolhida e optimizada para a previsão do estado das baterias, dada a sua capacidade de detetar padrões multi-periódicos em dados operacionais de baterias através da deteção de períodos baseada na Transformada Rápida de Fourier e de novas técnicas de transformação 2D.


%
\bigskip

\endgroup			

\cleardoublepage

\renewcommand{\abstractname}{Abstract}
\markboth{\spacedlowsmallcaps{\abstractname}}{\spacedlowsmallcaps{\abstractname}}
\refstepcounter{dummy}
\addcontentsline{toc}{chapter}{\abstractname}


\begingroup
\let\clearpage\relax
\let\cleardoublepage\relax

\chapter*{Abstract}

This report provides a thorough overview and analysis of artificial intelligence-based methods for monitoring the health of batteries in the automotive and rail industries. The analysis focuses on estimating battery state of health (SOH), state of charge (SOC) and remaining useful life (RUL) using advanced neural network architectures. It describes the entire battery health prediction process, from data preprocessing to the implementation of deep learning models for multi-parameter estimation. Various methodological approaches are reviewed, ranging from traditional physics-based models to state-of-the-art time series analysis architectures.

A evaluation of different neural network architectures is conducted, including a comparative analysis of Transformer and Mixture of Experts (MoE) networks. Based on these findings, the TimesNet architecture was chosen and optimised for battery health prediction, given its capacity to detect multi-periodic patterns in operational battery data via Fast Fourier Transform-based period detection and novel 2D transformation techniques.

%This development demonstrates improved performance compared to traditional battery monitoring methods.

\bigskip

\endgroup