%*******************************************************
% Abstract
%*******************************************************


\renewcommand{\abstractname}{Resumo}
\markboth{\spacedlowsmallcaps{\abstractname}}{\spacedlowsmallcaps{\abstractname}}
\refstepcounter{dummy}
\addcontentsline{toc}{chapter}{\abstractname}


\begingroup
\let\clearpage\relax
\let\cleardoublepage\relax

\chapter*{Resumo}
\color{Red}
A estimativa precisa do Estado de Saúde (SoH), Estado de Carga (SoC) e Vida Útil Restante (RUL) das baterias é crucial para aplicações automóveis e ferroviárias, dado o papel essencial das baterias na eficiência energética e fiabilidade dos sistemas de transporte.
A gestão eficaz destes parâmetros pode prevenir falhas inesperadas, otimizar os ciclos de carga e descarga, e prolongar a vida útil das baterias, contribuindo assim para uma redução significativa nos custos operacionais e ambientais. 
A Inteligência Artificial (IA) tem mostrado grande potencial na tarefa de estimar SoH, SoC e RUL das baterias. Algoritmos de machine learning e redes neuronais podem analisar grandes volumes de dados históricos e em tempo real, identificando padrões complexos que são difíceis de detectar com métodos tradicionais. 
A aplicação de IA permite uma previsão mais precisa e adaptativa das condições da bateria, melhorando a segurança e a eficiência operacional em veículos automóveis e ferroviários. Os datasets utilizados para a estimativa de SoH, SoC e RUL de baterias incluem uma variedade de dados recolhidos de ciclos de carga e descarga, condições de temperatura, tensões, correntes e outros parâmetros relevantes. 
Estes dados podem ser obtidos a partir de testes laboratoriais controlados, bem como de operações reais em campo. A qualidade e a abrangência dos datasets são essenciais para o treino eficaz dos modelos de IA, garantindo que eles possam generalizar bem para diferentes tipos de baterias e condições de operação. O desenvolvimento deste projeto envolve várias etapas-chave. 
Inicialmente, serão identificados os datasets e pré-processados os dados relevantes das baterias. Em seguida, serão desenvolvidos e treinados modelos de IA utilizando técnicas de machine learning supervisionado e não supervisionado. A validação dos modelos será realizada através de testes exaustivos com datasets distintos, assegurando a sua robustez e precisão. 
Finalmente, será implementado um sistema protótipo capaz de estimar em tempo real o SoH, SoC e RUL das baterias, com o objetivo de ser integrado em aplicações automóveis e ferroviárias, promovendo a inovação e a sustentabilidade nos sistemas de transporte.
\color{Black}


\bigskip

\endgroup			

\cleardoublepage

\renewcommand{\abstractname}{Abstract}
\markboth{\spacedlowsmallcaps{\abstractname}}{\spacedlowsmallcaps{\abstractname}}
\refstepcounter{dummy}
\addcontentsline{toc}{chapter}{\abstractname}


\begingroup
\let\clearpage\relax
\let\cleardoublepage\relax

\chapter*{Abstract}
...\\

\bigskip

\endgroup