%************************************************
\chapter{State of the Art}
\label{ch:stateoftheart}
%************************************************
\color{Red}comentário geral. Podia densificar um pouco mais e acrescentar figuras/tabelas (mesmo que sejam dos artigos que refere). Basta colocar a referencia na legenda. Ficava bem uma figura com a taxonomia  dos metodos\color{Black}

Getting the right predictions for SoC, SoH, and RUL is very important for making batteries work better in cars and trains. These predictions help make battery management systems (BMS) more reliable and work better. They provide key information for watching battery health and planning when to fix or replace batteries. Traditional methods, like Coulomb Counting and Kalman Filters, often have trouble with complex battery behavior and changing conditions. New advances in Artificial Intelligence (AI), especially machine learning and deep learning, provide better solutions by finding complex patterns in battery data. This chapter looks at the current best methods for SoC, SoH, and RUL estimation, focusing on AI-based approaches.

\section{Battery State Estimation}

\subsection{Traditional Methods}
\label{subsec:traditional_methods}
Traditional methods for checking battery state can be put into two groups: physics-based and statistical approaches, each with built-in problems.

\textbf{Physics-Based Methods} create models of how batteries work chemically and electrically. Key approaches include:
\begin{itemize}
    \item \textbf{Equivalent Circuit Models (ECMs)}: Show batteries using electrical parts (like resistors, capacitors) to copy voltage and current behavior. ECMs are fast to compute, but they are not very accurate when conditions change.\color{Red}faltam referências nesta subsecção\color{Black}
    \item \textbf{Electrochemical Models}: Copy internal chemical reactions, giving high accuracy but needing a lot of computer power and detailed knowledge of battery parameters.\color{Red}faltam referências nesta subsecção\color{Black}
\end{itemize}

Unlike these, \textbf{ statistical methods} use real data to estimate battery states. Common methods include:
\begin{itemize}
    \item \textbf{Coulomb Counting}: This method estimates SoC by integrating current over time, following the fundamental principle that charge accumulation equals the integral of current. While conceptually straightforward, coulomb counting suffers from significant practical limitations including sensitivity to measurement noise, current sensor drift, and initialization errors. Figure~\ref{fig:current_integration_error} illustrates how current integration errors accumulate over time, demonstrating the inherent challenges of this approach. The method's accuracy degrades particularly under varying sampling rates and temperature conditions~\cite{noauthor_implementation_nodate}. Despite these limitations, coulomb counting remains widely used due to its computational simplicity, often serving as a baseline method or being combined with other estimation techniques for improved reliability~\cite{movassagh_critical_2021}.

\begin{figure}[htbp]
\centering
\includegraphics[width=0.8\textwidth]{imgs/coulomb.jpg}
\caption{Current integration error accumulation over time showing how measurement uncertainties and sampling effects impact coulomb counting accuracy. The step-like behavior demonstrates the discrete nature of current sampling, while the smooth curve represents the theoretical continuous integration~\cite{movassagh_critical_2021}.}
\label{fig:current_integration_error}
\end{figure}

    \item \textbf{Kalman Filters}: Kalman filters represent a sophisticated recursive approach to state estimation that optimally combines model predictions with noisy measurements using statistical principles. The filter operates in two phases: prediction (using system dynamics) and correction (incorporating new measurements). For battery applications, the Extended Kalman Filter (EKF) is commonly employed to handle the nonlinear relationship between battery states and observable quantities such as terminal voltage~\cite{mastali_battery_2013}. Figure~\ref{fig:ekf_diagram} illustrates the EKF estimation process, showing how the algorithm iteratively refines state estimates by balancing model predictions with measurement data. While Kalman filters provide optimal estimates under Gaussian noise assumptions, their performance degrades significantly when dealing with the complex nonlinear dynamics and time-varying parameters characteristic of battery systems. The method requires accurate system models and proper tuning of noise covariance matrices, which can be challenging in practical applications where battery parameters drift over time due to aging and temperature variations~\cite{becker_wwwkalmanfilternet_online_nodate}.

\begin{figure}[htbp]
\centering
\includegraphics[width=0.9\textwidth]{imgs/ekf.png}
\caption{Extended Kalman Filter (EKF) State estimation process showing Prior Estimate (orange), Measurement (green), State Update (gray), and Current State Estimate (blue).~\cite{becker_wwwkalmanfilternet_online_nodate}.}
\label{fig:ekf_diagram}
\end{figure}
    
\end{itemize}

Because these methods often fail to capture small changes in battery behavior when conditions change, multiple AI-based approaches have been proposed in the last decade.

\subsection{AI-Based Methods}
\label{subsec:ai_methods}
AI-based methods use machine learning and deep learning to model complex relationships in battery data. This section looks at key approaches, datasets, and how they're used in this project.

\textbf{Machine Learning} algorithms that learn from examples, such as Support Vector Machines (SVMs) and Random Forests, have been used to predict SoC and SoH using features like voltage, current, and temperature. For example, \cite{sun_simultaneous_2022} shows SVMs getting high accuracy in SoH estimation (98.26\%) for lead-acid batteries under controlled conditions. However, these methods need a lot of feature engineering and have trouble with time-based patterns. On the contrary, \textbf{Deep Learning} models are very good at finding time-based and spatial patterns in battery data. Key types include:
\begin{itemize}
    \item \textbf{Convolutional Neural Networks (CNNs)}: Find spatial features from battery data, such as voltage profiles. Combining CNNs with Long Short-Term Memory (LSTM) units, as done in the \cite{Fangfang_Yang} paper, makes forecasting more accurate by modeling time-based patterns.
    \item \textbf{Recurrent Neural Networks (RNNs) and LSTMs}: Made for sequential data, LSTMs are very good for RUL prediction, as they capture long-term battery wear trends. Studies using the NASA Battery Dataset \cite{noauthor_nasa_nodate} show LSTM-based models work better than older methods in RUL estimation as done in the \cite{hong_state--health_2023}.
    \item \textbf{Transformer Models}: New in battery state estimation, transformers use attention mechanisms to model complex dependencies, showing promise in handling different-length sequences \cite{yilmaz_transformer-based_2025}.
\end{itemize}

\subsection{Hybrid Approaches}
\color{Red}faltam referências nesta subsecção\color{Black}
Hybrid models combine physics-based and data-driven methods to make results more accurate. For example, some studies combine ECMs with neural networks to improve SoC estimates, using physical constraints to reduce the amount of training data needed. Such approaches are very useful for railway applications, where operating conditions change a lot. Despite improvements, several challenges still exist in battery state estimation:
\begin{itemize}
    \item \textbf{Data Requirements}: AI models, especially deep learning, need large, varied datasets, which are often limited or private.
    \item \textbf{Operating Variability}: Battery performance changes due to temperature, load profiles, and aging, making it hard for models to work in different situations.
    \item \textbf{Computer Complexity}: Real-time estimation in cars and trains needs fast models, which is a challenge for complex deep learning systems.
    \item \textbf{Lack of Standard Datasets}: The absence of universal, open-source datasets for railway applications limits model comparison and testing.
\end{itemize}


\section{Datasets for AI-Based Estimation}
\label{subsec:datasets}
The quality and variety of datasets are very important for training strong AI models. A comprehensive analysis of available battery datasets reveals significant variation in data completeness, experimental conditions, and feature availability. Table~\ref{tab:battery_datasets_comparison} presents a detailed comparison of notable publicly available battery datasets, highlighting their characteristics and available measurements.

\begin{table}[htbp]
\centering
\caption{Comprehensive comparison of available battery datasets and their features}
\label{tab:battery_datasets_comparison}
\resizebox{\textwidth}{!}{%
\begin{tabular}{lcccccccccccccccc}
\hline
\textbf{Dataset} & \textbf{Cell Type} & \textbf{Chemistry} & \textbf{Time} & \textbf{C/D Ind.} & \textbf{Cycle} & \textbf{Current} & \textbf{Voltage} & \textbf{Dis. Cap.} & \textbf{Ch. Cap.} & \textbf{Ch. Energy} & \textbf{Dis. Energy} & \textbf{dV/dt} & \textbf{Int. Res.} & \textbf{AC Imp.} & \textbf{ACI Phase} & \textbf{Temp.} \\
\hline
Zn-ion, Na-ion @2025 & Various & Zn-ion, Na-ion & $\checkmark$ & $\checkmark$ & & $\checkmark$ & $\checkmark$ & $\checkmark$ & $\checkmark$ & & & & & & & \\
CALCE CS2 @2010 & Prismatic & LiCoO2 & $\checkmark$ & $\checkmark$ & $\checkmark$ & $\checkmark$ & $\checkmark$ & $\checkmark$ & $\checkmark$ & $\checkmark$ & $\checkmark$ & $\checkmark$ & $\checkmark$ & $\pm$ & $\pm$ & Initial \\
MATR @2019 & 18650 & LFP/graphite & $\checkmark$ & $\checkmark$ & $\checkmark$ & $\checkmark$ & $\checkmark$ & $\checkmark$ & $\checkmark$ & $\checkmark$ & $\checkmark$ & $\checkmark$ & $\checkmark$ & & & $\checkmark$ \\
MATR @2019 CL & 18650 & LFP/graphite & $\checkmark$ & & $\checkmark$ & $\checkmark$ & $\checkmark$ & $\checkmark$ & $\checkmark$ & $\checkmark$ & $\checkmark$ & $\checkmark$ & & & & $\checkmark$ \\
HUST @2022 & Various & LFP/graphite & $\checkmark$ & $\pm$ & $\checkmark$ & $\checkmark$ & $\checkmark$ & $\checkmark$ & $\checkmark$ & & & & & & & \\
RWTH @2017 & 18650 & Lithium Ion & $\checkmark$ & $\checkmark$ & $\checkmark$ & $\checkmark$ & $\checkmark$ & $\checkmark$ & $\checkmark$ & $\checkmark$ & $\checkmark$ & & & & & $\checkmark$ \\
ISU-ILCC @2023 & 502030 & Li-polymer & $\checkmark$ & & & $\checkmark$ & $\checkmark$ & $\checkmark$ & $\checkmark$ & $\checkmark$ & $\checkmark$ & & & & & \\
XJTU @2022 & 18650 & NCM Li-ion & $\checkmark$ & & & $\checkmark$ & $\checkmark$ & $\checkmark$ & $\checkmark$ & & & & & & & $\checkmark$ \\
Tongji @2022 & 18650 & NCA/NCM & $\checkmark$ & & $\checkmark$ & $\checkmark$ & $\checkmark$ & $\checkmark$ & $\checkmark$ & & & & & & & \\
Stanford @2024 & 21700 & Graphite/Si & $\checkmark$ & & $\checkmark$ & $\checkmark$ & $\checkmark$ & $\checkmark$ & $\checkmark$ & $\checkmark$ & $\checkmark$ & $\checkmark$ & $\checkmark$ & & & $\checkmark$ \\
\hline
\end{tabular}}
\end{table}

The analysis reveals several key insights about the current state of battery datasets:

Notable datasets with comprehensive feature sets include:
\begin{itemize}
    \item \textbf{NASA Battery Dataset} \cite{noauthor_nasa_nodate}: Provides voltage, current, temperature, and impedance data under different operating conditions, widely used for SoC and RUL estimation due to its diverse experimental scenarios and comprehensive measurement suite.
    \item \textbf{CALCE Battery Dataset} \cite{CALCE_battery_nodate}: Contains extensive aging data from lithium-ion batteries under different stress conditions, particularly valuable for SoH estimation and understanding battery degradation patterns. Notable for its complete feature set including energy measurements and impedance data.
    \item \textbf{MATR Battery Dataset} \cite{MATR_dataset_nodate}: Provides high-quality data from automotive battery testing, focusing on real-world driving conditions and temperature variations with comprehensive measurement capabilities.
    \item \textbf{Stanford Dataset}: Offers the most recent data with advanced battery chemistries (graphite/silicon) and comprehensive measurements including internal resistance and energy metrics.
\end{itemize}
These datasets show the importance of including real-world operating conditions and diagnostic measurements to make models work better in different situations.

%The \textit{BattAIHealth} project addresses these gaps by developing AI models trained on both synthetic and real-world datasets, making them work better and be suitable for real-time use in transportation systems.

\section{Discussion}
The current best methods in battery state estimation show a move from older physics-based and statistical methods to AI-driven approaches. While machine learning and deep learning models, supported by datasets like the NASA Battery Dataset and Aging Dataset from EV, give better accuracy, challenges such as limited data and changing operating conditions still exist. This project builds on these improvements by developing strong AI models made for car and train applications, aiming to make BMS more reliable and work better.