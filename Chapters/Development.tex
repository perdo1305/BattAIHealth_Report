%************************************************
\chapter{Development}
\label{ch:Development}
%************************************************
\lipsum[1]
\section{Dataset Collection and Preprocessing}

o dataset principal utilizado neste trabalho é o dataset //calce, que foi optido a partir do site x, da maneira xxxxx
falar da separacao do dataset
limpeza de dados (foi necessario remover o primeiro ciclo de cada ficheiro para  evitar problemas de inicializacao, e tambem remover os ciclos que nao tinham dados suficientes para serem utilizados)
como é feito o input dos dados para o modelo





\section{Utilized Model}
o modelo utilizado é uma arquitetura open source de redes neuronais, especialistga em dados temporais, esta biliotecam, times net é uma das que tem mais accuracy em termos de estimar dados temporais
.... falar um pouco da arquitetura, e como foi utilizado, e o que foi feito da minha parte

\section{Model Opimization}
para a optimizacao do modelo foi utilizada a ferramenta optuna, que permite a optimizacao de hiperparametros de modelos de machine learning, e a sua integracao com o wandb, que permite a visualizacao dos resultados e comparacao entre os modelos
falar dos parametros que foram otimizados, e como foi feita a optimizacao, e o que foi aprendido com isso
falar dos parametros mais importantes
