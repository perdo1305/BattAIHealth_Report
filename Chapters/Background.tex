%************************************************
\chapter{Background material and Supporting Technologies}
\label{ch:background}
%************************************************

This chapter presents the foundational background material and supporting technologies that were used in the project. The first section introduces the analytical frameworks used for battery data processing, including time series and spectral analysis techniques that are essential for understanding temporal patterns in battery behavior. Following this, we describe the comprehensive suite of software tools and platforms that facilitated the research and development process, from hyperparameter optimization and experiment tracking to data visualization and version control. The chapter then transitions to the core theoretical concepts fundamental to battery health monitoring, including detailed explanations of key battery parameters such as State of Charge (SoC), State of Health (SoH), and Remaining Useful Life (RUL). Additionally, we cover the essential evaluation metrics used to assess model performance and provide an in-depth discussion of battery degradation mechanisms that directly impact health estimation accuracy. Finally, the chapter addresses the technical challenges inherent in battery health monitoring, from the complexity of electrochemical processes to the practical difficulties of real-world implementation. This comprehensive background establishes the technical foundation necessary for understanding the methodologies and results presented in subsequent chapters.

\section{Time Series Analysis of Battery Data}
Time series analysis focuses on the temporal evolution of battery parameters such as voltage, current, and SoC. This approach models and forecasts battery behavior, identifies trends, and detects anomalies in the time domain.

\section{Spectral Analysis of Battery Data}
Spectral analysis is a frequency-domain technique used to characterize the dynamic behavior of battery systems by analyzing signals such as voltage, current, or impedance. This approach decomposes time-series data into its frequency components, revealing periodic patterns, noise characteristics, or system resonances that may not be evident in the time domain. Spectral analysis is particularly useful for understanding battery degradation, thermal effects, and electrochemical processes.

\section{Optuna}
Optuna is an open-source hyperparameter optimization framework used to search for the best hyperparameters in machine learning models \cite{akiba_optuna_2019}. It employs efficient algorithms like Tree-structured Parzen Estimator (TPE) to systematically explore hyperparameter spaces, supporting parallel and distributed optimization. In this work, Optuna was utilized to automate the tuning process, enhancing model performance by identifying optimal hyperparameter configurations with reduced manual effort.

\section{Weights and Biases (WandB)}
Weights \& Biases (WandB) is a machine learning platform designed for experiment tracking and visualization \cite{noauthor_weights_nodate}. It enables real-time logging and monitoring of training metrics, hyperparameters, and model outputs. In this study, WandB was employed to keep track of training processes and visualize losses, providing interactive dashboards to analyze experiments, debug models, and ensure reproducible workflows across iterations.

\section{PlotJuggler}
PlotJuggler is an open-source time series visualization tool designed for fast, intuitive, and extensible data analysis \cite{faconti_facontidavideplotjuggler_2025}. It features a user-friendly drag-and-drop interface, enabling efficient visualization of large datasets, both offline and in real-time. In this work, PlotJuggler was highly effective for exploring and analyzing data within datasets, allowing for the visualization of time series, identification of patterns, and application of transformations like derivatives or moving averages through its Transform Editor. Its compatibility with formats such as CSV, ROS, and MQTT, along with plugin extensibility, made it a valuable tool for detailed data inspection and analysis.

\section{Orange Data Mining}
Orange Data Mining is an open-source data visualization and analysis platform designed for exploratory data analysis and machine learning workflows \cite{noauthor_biolaborange3_nodate}. It provides a visual programming interface with drag-and-drop widgets that enable users to build data analysis pipelines without extensive coding. Orange offers comprehensive tools for data preprocessing, feature selection, correlation analysis, and outlier detection through interactive visualizations and statistical methods. In this work, Orange was instrumental for exploring correlations within battery datasets and identifying outliers that could potentially skew model performance. Its intuitive interface allowed for rapid prototyping of data cleaning pipelines, enabling efficient removal of anomalous data points while preserving the integrity of temporal relationships in battery time series data. The platform's correlation analysis capabilities were particularly valuable for understanding feature relationships and selecting relevant variables for subsequent machine learning models.

\section{Git Version Control}
Git is a distributed version control system designed to handle projects of all sizes with speed and efficiency \cite{noauthor_git_nodate}. It tracks changes in source code and files during software development, enabling multiple contributors to work on the same project simultaneously while maintaining a complete history of modifications. Git provides features such as branching, merging, and distributed workflows that facilitate collaborative development and experimentation. In this work, Git was employed to ensure robust version control throughout the research process, maintaining a comprehensive history of code changes, experimental iterations, and documentation updates. All project files, including machine learning models, data processing scripts, and analysis notebooks, were systematically committed and pushed to GitHub repositories. This approach guaranteed that all work was safely preserved, enabled rollback to previous versions when needed, and facilitated reproducible research by maintaining clear documentation of the project's evolution and experimental milestones.

\section{PyTorch}
PyTorch is an open-source machine learning framework developed by Facebook's AI Research lab, designed for deep learning applications with a focus on flexibility and ease of use \cite{ansel_pytorch_2024}. It provides dynamic computational graphs, allowing for intuitive model development and debugging through its eager execution model. PyTorch features automatic differentiation capabilities through its autograd system, enabling efficient gradient computation for backpropagation in neural networks. The framework supports GPU acceleration through CUDA, making it suitable for training large-scale models efficiently. In this work, PyTorch served as the primary framework for developing and training deep learning models for battery health estimation. Its dynamic nature was particularly valuable for experimenting with different neural network architectures, including LSTM networks for temporal sequence modeling and CNN-LSTM hybrid models for battery state prediction. The framework's extensive ecosystem, including libraries like PyTorch Lightning for structured training loops and TorchVision for data processing utilities, facilitated rapid prototyping and model development. PyTorch's seamless integration with other tools in the machine learning pipeline, such as Optuna for hyperparameter optimization and WandB for experiment tracking, made it an ideal choice for this research project's requirements.

\section{Conda environments}
Conda is an open-source package management and environment management system that simplifies the installation, running, and updating of packages and their dependencies \cite{conda_contributors_conda_2025}. It creates isolated environments where different versions of Python, libraries, and dependencies can coexist without conflicts, making it particularly valuable for research projects requiring specific software configurations. Conda supports multiple programming languages including Python, R, Ruby, Lua, Scala, Java, JavaScript, C/C++, and FORTRAN, and can install packages from various repositories including conda-forge, PyPI, and custom channels. In this work, Conda was essential for maintaining reproducible development environments across different stages of the battery health estimation project. Separate conda environments were created for different aspects of the research: one environment for data preprocessing and exploratory analysis with packages like pandas, numpy, and matplotlib; another for deep learning model development with PyTorch, CUDA libraries, and related dependencies; and a third for experiment tracking and visualization with WandB, Optuna, and Jupyter notebooks. This approach ensured that version conflicts between packages were avoided, enabled seamless collaboration across different development machines, and guaranteed that the exact software environment could be recreated for reproducibility. The use of environment.yml files allowed for easy sharing and recreation of these environments, supporting the project's open science principles and facilitating future research extensions.

\section{Core Concepts}
Battery technology serves as the foundation for energy storage systems across numerous applications, from portable electronics to electric vehicles and grid-scale storage. Modern batteries primarily fall into several categories, including lithium-ion, lead-acid, nickel-metal hydride, and flow batteries, each with distinct electrochemical properties, energy densities, and lifecycle characteristics. The health of these batteries is characterized by parameters such as state of charge (SoC), state of health (SoH), capacity fade, internal resistance, and degradation rates, which collectively determine performance and longevity. Monitoring these parameters presents unique challenges due to the complex, nonlinear relationships between observable measurements and underlying battery conditions. Artificial intelligence and machine learning approaches offer powerful solutions to these challenges by enabling pattern recognition across multidimensional battery data. Supervised learning algorithms can predict remaining useful life, while unsupervised methods can detect anomalies indicative of impending failure. Deep learning architectures, particularly recurrent neural networks and transformers, have demonstrated exceptional capability in extracting temporal patterns from battery operational data, making them especially valuable for health prognostics in dynamic usage scenarios.

%Battery fundamentals (types, chemistry, operating principles)
%Key health monitoring parameters for batteries
%AI/ML fundamentals relevant to your application

\subsection{State of Charge (SoC)}
The State of Charge represents the amount of energy remaining in a battery relative to its maximum capacity. It can be expressed as:
\[
\text{SoC} = \frac{\text{Remaining Charge or Energy}}{\text{Maximum Charge or Energy Capacity}} \times 100\%
\]
However, due to the chemical complexity of batteries and variations among individual cells, the SoC is always an approximate estimate. One factor contributing to the nonlinearity in its estimation is the formation of impurity layers in the pores of the electrodes. When these pores are blocked by impurities, electron movement is hindered, leading to irregular voltages and currents.

% Creating section for State of Health
\subsection{State of Health (SoH)}
The State of Health reflects the battery's ability to store and deliver energy compared to its original specifications. It can be expressed as:
\[
\text{SoH} = \frac{\text{Current Maximum Capacity}}{\text{Original Maximum Capacity}} \times 100\%
\]
The nonlinearity in SoH estimation primarily arises from the progressive degradation of electrode materials. As impurities accumulate in the electrode pores, the available surface area for electrochemical reactions decreases, reducing the battery's effective capacity. This process is highly dependent on the number of charge/discharge cycles and operational conditions, making it difficult to model SoH linearly over time.

% Creating section for Remaining Useful Life
\subsection{Remaining Useful Life (RUL)}
The Remaining Useful Life quantifies the number of cycles remaining before the battery's performance degrades below a specified threshold. It can be expressed as:
\[
\text{RUL} = \text{Total Expected Useful Life} - \text{Current Age}
\]
Predicting RUL is particularly challenging due to the accumulation of impurities in the electrode pores. As the pores become obstructed, the degradation rate accelerates, leading to a sudden drop in battery performance. This nonlinear dynamic makes it difficult to accurately predict the exact point at which the battery will reach its end of useful life.
% Explaining Mean Absolute Error
\subsection{Mean Absolute Error (MAE)}
The Mean Absolute Error measures the average magnitude of errors in a set of predictions, without considering their direction. It is calculated as:
\begin{equation}
\text{MAE} = \frac{1}{n} \sum_{i=1}^{n} |y_i - \hat{y}_i|
\end{equation}
where $y_i$ is the actual value, $\hat{y}_i$ is the predicted value, and $n$ is the number of observations. MAE is intuitive and robust to outliers, as it does not square the errors, but it does not penalize larger errors as heavily as other metrics. This makes it less sensitive to extreme deviations in predictions, which can be a limitation in contexts like battery performance where large errors may indicate critical failures.

% Explaining Mean Squared Error
\subsection{Mean Squared Error (MSE)}
The Mean Squared Error quantifies the average of the squared differences between predicted and actual values. It is expressed as:
\begin{equation}
\text{MSE} = \frac{1}{n} \sum_{i=1}^{n} (y_i - \hat{y}_i)^2
\end{equation}
MSE emphasizes larger errors due to the squaring of differences, making it sensitive to outliers. In battery modeling, this can be useful for detecting significant deviations in predictions of parameters like State of Charge or State of Health, but its sensitivity to outliers may amplify the impact of irregular data points caused by factors like electrode impurities or sensor noise.

% Explaining Root Mean Squared Error
\subsection{Root Mean Squared Error (RMSE)}
The Root Mean Squared Error is the square root of the MSE, providing an error metric in the same units as the original data. It is defined as:
\begin{equation}
\text{RMSE} = \sqrt{\frac{1}{n} \sum_{i=1}^{n} (y_i - \hat{y}_i)^2}
\end{equation}
RMSE balances the emphasis on larger errors from MSE while being more interpretable due to its unit consistency with the data. In battery applications, RMSE is often used to evaluate prediction accuracy for metrics like SoC or RUL, but its sensitivity to outliers can be a drawback when dealing with nonlinear degradation patterns caused by electrode pore blockages.

% Explaining Mean Absolute Percentage Error
\subsection{Mean Absolute Percentage Error (MAPE)}
The Mean Absolute Percentage Error measures the average percentage error between predicted and actual values. It is calculated as:
\begin{equation}
\text{MAPE} = \frac{1}{n} \sum_{i=1}^{n} \left| \frac{y_i - \hat{y}_i}{y_i} \right| \times 100\%
\end{equation}
MAPE is useful for comparing prediction accuracy across datasets with different scales, as it normalizes errors relative to the actual values. However, it can become problematic when actual values are close to zero, as in some battery SoC scenarios, leading to large percentage errors. Additionally, its reliance on relative errors may mask significant absolute deviations in critical battery performance metrics.

\subsection{Battery Degradation}
Battery degradation refers to the gradual loss of a battery's ability to store and deliver
energy, driven by chemical reactions, temperature fluctuations, charge/discharge cycles,
and aging. The more important mechanisms include:

\begin{itemize}
    \item \textbf{Solid Electrolyte Interphase (SEI) Growth}: A layer forms on the anode, consuming lithium ions and reducing capacity. This process is accelerated at high temperatures and currents.
    \item \textbf{Lithium Plating}: At low temperatures or high charge rates, lithium deposits on the anode, forming ``dead lithium'' that contributes to irreversible capacity loss.
    \item \textbf{Particle Fracture}: Mechanical stress from cycling causes cracks in electrode materials, reducing active material availability.
    \item \textbf{Positive Electrode (PE) Decomposition}: Structural changes in the cathode, such as spinel/rock salt phase formation, degrade performance.
\end{itemize}

\subsection{Capacity Fade}
The \cite{zhang_studies_2000} paper studies and explains very well the capacity fade refers to the progressive decline in a lithium-ion battery’s ability to store energy, manifesting as reduced device runtime or diminished electric vehicle driving range. This phenomenon is driven by several key mechanisms:

\begin{itemize}
\item \textbf{Loss of Lithium Inventory (LLI)}: Growth of the solid electrolyte interphase (SEI) on the carbon anode consumes cyclable lithium, leading to an initial irreversible capacity loss of approximately 10\% during formation cycles. Subsequent SEI growth further reduces capacity over time.
\item \textbf{Loss of Active Material (LAM)}: Particle fracture, phase changes, and positive electrode (PE) decomposition, degrade active material availability, exacerbating capacity decline.
\item \textbf{Lithium Plating}: At high SoC or low temperatures, lithium deposition forms ``dead lithium,'' causing irreversible capacity loss and increasing safety risks.
\item \textbf{Impedance Increase}: Rising interfacial resistance, predominantly at the positive electrode, limits efficient charge transfer, indirectly reducing usable capacity. After 800 cycles, the electrode resistance can increase tenfold, significantly contributing to capacity fade.
\end{itemize}

The study reports capacity losses ranging from 12.4\% to 32\% after 500–800 cycles, corresponding to an average loss of 0.025–0.05\% per cycle.


\subsection{Internal Resistance Degradation in Lithium-Ion Batteries}

As lithium-ion batteries age, their internal resistance increases, adversely affecting power delivery, charging efficiency, and thermal management. This degradation is particularly pronounced during calendar ageing, as detailed in the study by \cite{stroe_degradation_2018} on LFP/C-based batteries. The primary mechanisms contributing to this increase include:

\begin{itemize}
\item \textbf{Solid Electrolyte Interface (SEI) Growth}: The SEI layer on the graphite anode thickens over time, reducing Li$^+$ ion permeability. This growth follows a power law dependence (approximately $t^{0.8}$) and is accelerated at high temperatures and high state-of-charge (SOC) levels, leading to increased resistance and contact loss within the anode.
\item \textbf{Lithium Plating}: Deposition of metallic lithium on the anode clogs electrode pores, impeding ion transport and elevating resistance, particularly under high SOC conditions.
\item \textbf{Cathode Structural Degradation}: At the LFP cathode, binder decomposition, oxidation of conductive agents, and corrosion of current collectors reduce inter-particle conductivity, contributing to resistance increase, especially at elevated temperatures.
\item \textbf{Electrolyte Decomposition}: Decomposition products form resistive surface layers on both electrodes, further increasing internal resistance, with effects amplified at high temperatures and SOC levels.
\end{itemize}

The study demonstrates that internal resistance increases nonlinearly with storage time, with exponential acceleration due to higher storage temperatures (e.g., 55°C) and SOC levels (e.g., 90\%). For instance, after 20 years at 25°C and 50\% SOC, resistance may rise by approximately 71\%, doubling at 100\% SOC. This increased resistance results in slower charging, diminished power output, and accelerated degradation due to enhanced heat generation, impacting battery performance and lifespan.

\subsection{Battery Health Monitoring}
Battery health monitoring is critical for ensuring reliability, safety, and longevity of battery systems. Monitoring involves assessing key parameters such as the state of charge and the state of health (SoH), which provide essential insights into battery performance and remaining operational capacity.
% Technical Challenges section
\subsubsection{Technical Challenges}
The technical challenges in monitoring battery health arise from the complex nature of battery systems and the difficulties in accurately estimating SOC and SOH.

\subsubsection{Complexity of Battery Chemistry}
Batteries, particularly lithium-ion batteries, have intricate internal chemistries that are difficult to model and monitor.
Factors such as temperature, charge-discharge rates, and depth of discharge influence degradation, making accurate SOH estimation challenging. 
The nonlinear and complex degradation processes vary with usage conditions, environmental factors, and battery design, complicating predictive modeling.

\subsubsection{Measurement Difficulties}
Measuring individual battery parameters, such as internal resistance, temperature, and voltage, is technically challenging, especially in real-time applications. 
This requires precise sensors and sophisticated equipment, which may not be feasible in real-world scenarios. 
For instance, accurately measuring internal resistance or temperature in a moving vehicle is far more complex than in a controlled lab environment.

\subsubsection{Modeling and Estimation}
\textbf{rever isto !!!} \\
Developing accurate models for SOH estimation is complex. 
Electrochemical models, which simulate battery behavior based on physical and chemical principles, require extensive computational resources and detailed parameter inputs (e.g., electrolyte properties, reaction rates). 
Semi-empirical models often oversimplify electrochemical processes, reducing their effectiveness under extreme conditions. Equivalent circuit models (ECMs) may lack precision during high-rate charging/discharging or extreme temperatures due to their simplified nature.

\subsubsection{Limitations of Data-Driven Methods}
Data-driven approaches, such as machine learning techniques (e.g. Support Vector Regression, Gaussian Process Regression, Artificial Neural Networks), rely on large, high-quality datasets, which can be difficult to obtain. 
These methods also lack physical interpretability, making it difficult to understand their predictions. 
Additionally, issues like overfitting and high computational demands pose challenges for real-time applications.

\subsubsection{Complexity of Hybrid Methods}
Hybrid approaches, which combine model-based and data-driven methods, can improve accuracy but increase system complexity and computational costs. 
Interpreting errors in these systems remains a challenge, requiring further research to enhance transparency and efficiency.

\subsection{Laboratory vs Real World Conditions}
There is a significant discrepancy between laboratory-simulated conditions and actual operational environments. 
Laboratory settings often use sophisticated equipment that is not available in real-world applications, limiting the applicability of monitoring methods. For example, real-world conditions like varying temperatures or road vibrations are difficult to replicate in a lab, affecting SOH estimation accuracy.

\subsection{Scalability}
Monitoring systems must handle large numbers of batteries, especially in applications like EVs or trains. 
This requires scalable solutions that can process vast amounts of data efficiently. 
For instance, managing battery health across a fleet of electric vehicles demands robust, centralized data systems that can scale without compromising performance.

\subsection{Real-Time Monitoring}
Achieving real-time, reliable SOH monitoring is crucial for safety-critical applications but is technically demanding. 
Battery management systems (BMS) must balance accuracy with computational efficiency to provide timely insights without overloading system resources.

\subsection{Environmental Factors}
Batteries are sensitive to environmental conditions such as temperature, humidity, and vibration. 
Monitoring systems must account for these factors, which can significantly impact battery health and performance. 
For example, high temperatures can accelerate battery degradation, while low temperatures may reduce capacity, complicating health estimation.

\subsection{Cost of Monitoring Systems}
Implementing sophisticated battery health monitoring systems can be expensive, both in terms of initial setup and ongoing maintenance. 
This includes the cost of sensors, data storage, and computational infrastructure, which can be prohibitive for smaller organizations or applications.

\subsection{Data and Computational Costs}
AI and data-driven methods require significant computational resources and high-quality data, which can be costly to acquire and process. 
The high demand for data and computing power presents challenges, particularly for real-time monitoring applications and edge devices.

\subsection{Standardization}
The lack of standardized methods and parameters for battery health monitoring hinders interoperability and comparison across different systems. 
This fragmentation makes it difficult to integrate monitoring solutions across diverse applications, such as consumer electronics and industrial systems.

%\section{Supporting Technologies}
%Sensor technologies for battery data collection
%Data acquisition systems and protocols
%Computing platforms/hardware used


%\section{Methodogical Background}
%Signal processing techniques for battery data
%Feature extraction methods
%Relevant machine learning algorithms (classification, regression, etc.)
%Evaluation metrics for health monitoring systems


%\section{Related Frameworks}
%Software libraries and tools used in implementation
%Data management approaches
%Visualization techniques
