\addtocontents{toc}{\protect\vspace{\beforebibskip}} % Place slightly below the rest of the document content in the table

%************************************************
\chapter{Introdução}
\label{ch:introduction}
%************************************************


Este documento serve de orientação para o relatório da unidade curricular de Projeto do Curso de Engenharia Eletrotécnica e de Computadores da ESTG – IPLEIRIA. Como tal, é constituído por um conjunto predefinido de estilos. Estes devem ser utilizados sem serem alterados ou substituídos. Para começar a escrever o relatório, basta guardar uma cópia deste documento e substituir os campos e as secções de acordo com o projeto em questão.

A intenção deste documento é fazer com que os estudantes se concentrem na produção dos conteúdos e não se preocupem com formatações de tipos de letra, parágrafos, etc.

Quanto à introdução, ela deve preparar o leitor para o resto do relatório.

Entre outros, deverá incluir:
\begin{itemize}
\item uma apresentação do assunto e contexto;
\item a definição do objetivo do trabalho;
\item as preocupações que o motivaram;
\item a metodologia de ação para a persecução do trabalho;
\item a colaboração solicitada e recebida e o grau de auxílio que esse eventual apoio teve para o bom andamento dos trabalhos;
\item eventualmente, aspetos particulares do trabalho: situações vividas e técnicas adquiridas.
\end{itemize}

Por fim, a introdução deve descrever como foi organizado o relatório, referindo, brevemente, o propósito de cada secção considerada no mesmo.

As referências bibliográficas são extremamente importantes e podem ser feitas da seguinte forma (ver código fonte do \LaTeX): 
\begin{itemize}
    \item Para fazer uma citação no fim de uma frase: \parencite{Sims1992}. 
    \item Múltiplas citações: \parencite{Darwin1859,Koza1992}.
    \item Para fazer uma citação que serve também como sujeito dessa frase (por exemplo no início): \textcite{Sims1992}.
    \item Obter apenas o nome do autor: \citeauthor{Sims1992}.
    \item Obter apenas o título do obra: \citetitle{Sims1992}.
    \item Segundo \textcite{Rudolph2016}, bla, bla .... \citetitle{Rudolph2016}.
\end{itemize}

Para mais informações sobre o \LaTeXe\ aconselha-se a consulta do livro \emph{The Not So Short Introduction to \LaTeXe} \parencite{oetiker2000nss}.

Para a gestão de referências bibliográficas aconselha-se o JabRef. %\parencite{jabref}.

o trabalho desenvolvido tem como objetivo estimar o estado de saude (SoH), estado de carga (SoC) e vida útil restante (RUL) de baterias, com foco em aplicações automóveis e ferroviárias. A gestão eficaz destes parâmetros é crucial para otimizar o desempenho e a durabilidade das baterias, prevenindo falhas inesperadas e melhorando a eficiência energética dos sistemas de transporte.
onde as tecnicas atuais nao conseguem ser precisas o suficiente, e a inteligencia artificial pode ajudar a melhorar a precisão das estimativas de SoH, SoC e RUL. A IA pode analisar grandes volumes de dados históricos e em tempo real, identificando padrões complexos que são difíceis de detectar com métodos tradicionais. A aplicação de IA permite uma previsão mais precisa e adaptativa das condições da bateria, melhorando a segurança e a eficiência operacional em veículos automóveis e ferroviários.
em todos os tipos de ambiente, temperaturas etc, e para diferentes quimixas de baterias.
isto serve de motivacao para a criacao de redes neuronais que consigam estimar isto tudo

para atacar este problema foi necessario adquirir conhecimento sobre baterias, quimica, funcionamento, etc.
bem como estrutura de varias arquiteturas de redes neuroais e tecnicas utilizadas para criar o modelo, avalia-lo, testa lo e utilizacao de grandes quantidades de dados
depois selecionar a arquitetura/biblioreca mais aduada para o problema, e neste caso treinar a rede neuronar e a valida-a optimiza la 

esta sendo uma area diversa ao lecionado no curso, tive ajuda e transferencia de conhecimento dos meus orientadores que me guiaram para a descoberta de novos conhecinentos e ferramneteas para ajudar no trabalho desenvolvido

adqui tecnicas e conhecimentos de manipulacao de dados, analise de dados, e utilizacao de redes neuronais para resolver problemas complexos
ferramentas de  optimizacao de modelos como o optina e visualizao como o wandb