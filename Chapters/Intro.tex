\addtocontents{toc}{\protect\vspace{\beforebibskip}} % Place slightly below the rest of the document content in the table

%************************************************
\chapter{Introdução}
\label{ch:introduction}
%************************************************


Este documento serve de orientação para o relatório da unidade curricular de Projeto do Curso de Engenharia Eletrotécnica e de Computadores da ESTG – IPLEIRIA. Como tal, é constituído por um conjunto predefinido de estilos. Estes devem ser utilizados sem serem alterados ou substituídos. Para começar a escrever o relatório, basta guardar uma cópia deste documento e substituir os campos e as secções de acordo com o projeto em questão.

A intenção deste documento é fazer com que os estudantes se concentrem na produção dos conteúdos e não se preocupem com formatações de tipos de letra, parágrafos, etc.

Quanto à introdução, ela deve preparar o leitor para o resto do relatório.

Entre outros, deverá incluir:
\begin{itemize}
\item uma apresentação do assunto e contexto;
\item a definição do objetivo do trabalho;
\item as preocupações que o motivaram;
\item a metodologia de ação para a persecução do trabalho;
\item a colaboração solicitada e recebida e o grau de auxílio que esse eventual apoio teve para o bom andamento dos trabalhos;
\item eventualmente, aspetos particulares do trabalho: situações vividas e técnicas adquiridas.
\end{itemize}

Por fim, a introdução deve descrever como foi organizado o relatório, referindo, brevemente, o propósito de cada secção considerada no mesmo.

As referências bibliográficas são extremamente importantes e podem ser feitas da seguinte forma (ver código fonte do \LaTeX): 
\begin{itemize}
    \item Para fazer uma citação no fim de uma frase: \parencite{Sims1992}. 
    \item Múltiplas citações: \parencite{Darwin1859,Koza1992}.
    \item Para fazer uma citação que serve também como sujeito dessa frase (por exemplo no início): \textcite{Sims1992}.
    \item Obter apenas o nome do autor: \citeauthor{Sims1992}.
    \item Obter apenas o título do obra: \citetitle{Sims1992}.
    \item Segundo \textcite{Rudolph2016}, bla, bla .... \citetitle{Rudolph2016}.
\end{itemize}

Para mais informações sobre o \LaTeXe\ aconselha-se a consulta do livro \emph{The Not So Short Introduction to \LaTeXe} \parencite{oetiker2000nss}.

Para a gestão de referências bibliográficas aconselha-se o JabRef. %\parencite{jabref}.