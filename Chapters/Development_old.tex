\addtocontents{toc}{\protect\vspace{\beforebibskip}} % Place slightly below the rest of the document content in the table



%************************************************
\chapter{Background material and Supporting Technologies}
\label{ch:background}
%************************************************
\lipsum[1]

\section{Time Series Analysis of Battery Data}
Time series analysis focuses on the temporal evolution of battery parameters such as voltage, current, and SoC. This approach models and forecasts battery behavior, identifies trends, and detects anomalies in the time domain.

\section{Spectral Analysis of Battery Data}
Spectral analysis is a frequency-domain technique used to characterize the dynamic behavior of battery systems by analyzing signals such as voltage, current, or impedance. This approach decomposes time-series data into its frequency components, revealing periodic patterns, noise characteristics, or system resonances that may not be evident in the time domain. Spectral analysis is particularly useful for understanding battery degradation, thermal effects, and electrochemical processes.

\section{Core Concepts}

Battery technology serves as the foundation for energy storage systems across numerous applications, from portable electronics to electric vehicles and grid-scale storage. Modern batteries primarily fall into several categories, including lithium-ion, lead-acid, nickel-metal hydride, and flow batteries, each with distinct electrochemical properties, energy densities, and lifecycle characteristics. The health of these batteries is characterized by parameters such as state of charge (SoC), state of health (SoH), capacity fade, internal resistance, and degradation rates, which collectively determine performance and longevity. Monitoring these parameters presents unique challenges due to the complex, nonlinear relationships between observable measurements and underlying battery conditions. Artificial intelligence and machine learning approaches offer powerful solutions to these challenges by enabling pattern recognition across multidimensional battery data. Supervised learning algorithms can predict remaining useful life, while unsupervised methods can detect anomalies indicative of impending failure. Deep learning architectures, particularly recurrent neural networks and transformers, have demonstrated exceptional capability in extracting temporal patterns from battery operational data, making them especially valuable for health prognostics in dynamic usage scenarios.

%Battery fundamentals (types, chemistry, operating principles)
%Key health monitoring parameters for batteries
%AI/ML fundamentals relevant to your application

\textbf{Battery Degradation}
Battery degradation refers to the gradual loss of a battery’s ability to store and deliver
energy, driven by chemical reactions, temperature fluctuations, charge/discharge cycles,
and aging. The more important mechanisms include:

\begin{itemize}
    \item \textbf{Solid Electrolyte Interphase (SEI) Growth}: A layer forms on the anode, consuming lithium ions and reducing capacity. This process is accelerated at high temperatures and currents.
    \item \textbf{Lithium Plating}: At low temperatures or high charge rates, lithium deposits on the anode, forming ``dead lithium'' that contributes to irreversible capacity loss.
    \item \textbf{Particle Fracture}: Mechanical stress from cycling causes cracks in electrode materials, reducing active material availability.
    \item \textbf{Positive Electrode (PE) Decomposition}: Structural changes in the cathode, such as spinel/rock salt phase formation, degrade performance.
\end{itemize}

\textbf{Capacitty Fade}
Capacity fade is the reduction in a battery's energy storage capacity, observed as shorter device runtimes or reduced EV driving ranges. Key contributors include:

\begin{itemize}
    \item \textbf{LLI}: SEI growth traps cyclable lithium, causing an initial capacity reduction of approximately 10\% .
    \item \textbf{LAM}: Particle fracture and PE decomposition disrupt active material, further reducing capacity.
    \item \textbf{Lithium Plating}: Forms ``dead lithium,'' contributing to irreversible capacity loss, especially at high state of charge (SoC) or low temperatures.
    \item \textbf{Impedance Increase}: Higher resistance indirectly reduces usable capacity by limiting efficient energy transfer.
\end{itemize}

\textbf{colocar citacao aqui !!!} \\
Studies report capacity loss in lithium-ion batteries ranging from 12.4\% to 24.1\% after 500 cycles, averaging 0.025--0.048\% per cycle.

\textbf{Internal Resistance Degradation}
Internal resistance increases as batteries age, impacting power delivery, charging speed, and heat generation. Key causes include:

\begin{itemize}
    \item \textbf{SEI Growth}: The SEI layer becomes less permeable to Li$^+$ ions, with thickness increasing with the square root of time (e.g., 15--100 nm for spinel/rock salt layers, ~10 nm for passivating SEI after one year of storage).
    \item \textbf{Lithium Plating}: Clogs electrode pores, increasing resistance.
    \item \textbf{PE Structural Changes}: Formation of spinel/rock salt phases and passivating SEI (pSEI) layers adds resistance.
    \item \textbf{Particle Fracture}: Disrupts electrical conductivity, further increasing resistance.
\end{itemize}
Increased resistance leads to slower charging, reduced performance, and accelerated degradation due to heat generation.


\textbf{monitoring battery health}
Battery health monitoring is critical for ensuring the reliability, safety, and longevity. Monitoring involves assessing the state of charge,
and the state of health (SOH),
como vimos no capitulo anterior....blablabla
% Technical Challenges section
\textbf{Technical Challenges}
The technical challenges in monitoring battery health arise from the complex nature of battery systems and the difficulties in accurately estimating SOC and SOH.

\textbf{Complexity of Battery Chemistry}
Batteries, particularly lithium-ion batteries, have intricate internal chemistries that are difficult to model and monitor.
Factors such as temperature, charge-discharge rates, and depth of discharge influence degradation, making accurate SOH estimation challenging. 
The nonlinear and complex degradation processes vary with usage conditions, environmental factors, and battery design, complicating predictive modeling.

\textbf{Measurement Difficulties}
Measuring individual battery parameters, such as internal resistance, temperature, and voltage, is technically challenging, especially in real-time applications. 
This requires precise sensors and sophisticated equipment, which may not be feasible in real-world scenarios. 
For instance, accurately measuring internal resistance or temperature in a moving vehicle is far more complex than in a controlled lab environment.

\textbf{Modeling and Estimation}
\textbf{rever isto !!!} \\
Developing accurate models for SOH estimation is complex. 
Electrochemical models, which simulate battery behavior based on physical and chemical principles, require extensive computational resources and detailed parameter inputs (e.g., electrolyte properties, reaction rates). 
Semi-empirical models often oversimplify electrochemical processes, reducing their effectiveness under extreme conditions. Equivalent circuit models (ECMs) may lack precision during high-rate charging/discharging or extreme temperatures due to their simplified nature.

\textbf{Limitations of Data-Driven Methods}
Data-driven approaches, such as machine learning techniques (e.g. Support Vector Regression, Gaussian Process Regression, Artificial Neural Networks), rely on large, high-quality datasets, which can be difficult to obtain. 
These methods also lack physical interpretability, making it difficult to understand their predictions. 
Additionally, issues like overfitting and high computational demands pose challenges for real-time applications.

\textbf{Complexity of Hybrid Methods}
Hybrid approaches, which combine model-based and data-driven methods, can improve accuracy but increase system complexity and computational costs. 
Interpreting errors in these systems remains a challenge, requiring further research to enhance transparency and efficiency.

% Practical Challenges section
\section{Practical Challenges}
Practical challenges stem from the gap between controlled laboratory environments and real-world operational conditions, as well as the need for scalable and reliable monitoring systems on edge devices.

\textbf{Laboratory vs. Real-World Conditions}
There is a significant discrepancy between laboratory-simulated conditions and actual operational environments. 
Laboratory settings often use sophisticated equipment that is not available in real-world applications, limiting the applicability of monitoring methods. For example, real-world conditions like varying temperatures or road vibrations are difficult to replicate in a lab, affecting SOH estimation accuracy.

\textbf{Scalability}
Monitoring systems must handle large numbers of batteries, especially in applications like EVs or trains. 
This requires scalable solutions that can process vast amounts of data efficiently. 
For instance, managing battery health across a fleet of electric vehicles demands robust, centralized data systems that can scale without compromising performance.

\textbf{Real-Time Monitoring}
Achieving real-time, reliable SOH monitoring is crucial for safety-critical applications but is technically demanding. 
Battery management systems (BMS) must balance accuracy with computational efficiency to provide timely insights without overloading system resources.

\textbf{Environmental Factors}
Batteries are sensitive to environmental conditions such as temperature, humidity, and vibration. 
Monitoring systems must account for these factors, which can significantly impact battery health and performance. 
For example, high temperatures can accelerate battery degradation, while low temperatures may reduce capacity, complicating health estimation.

% Economic and Feasibility Challenges section
\section{Economic and Feasibility Challenges}
Economic considerations and the feasibility of implementing advanced monitoring systems present additional hurdles.

\textbf{Cost of Monitoring Systems}
Implementing sophisticated battery health monitoring systems can be expensive, both in terms of initial setup and ongoing maintenance. 
This includes the cost of sensors, data storage, and computational infrastructure, which can be prohibitive for smaller organizations or applications.

\textbf{Data and Computational Costs}
AI and data-driven methods require significant computational resources and high-quality data, which can be costly to acquire and process. 
The high demand for data and computing power its a problem, particularly for real-time monitoring applications and edge devices.

\textbf{Standardization}
The lack of standardized methods and parameters for battery health monitoring hinders interoperability and comparison across different systems. 
This fragmentation makes it difficult to integrate monitoring solutions across diverse applications, such as consumer electronics and industrial systems.

%\section{Supporting Technologies}
%Sensor technologies for battery data collection
%Data acquisition systems and protocols
%Computing platforms/hardware used


%\section{Methodogical Background}
%Signal processing techniques for battery data
%Feature extraction methods
%Relevant machine learning algorithms (classification, regression, etc.)
%Evaluation metrics for health monitoring systems


%\section{Related Frameworks}
%Software libraries and tools used in implementation
%Data management approaches
%Visualization techniques

%************************************************
\chapter{State of the Art}
\label{ch:stateoftheart}
%************************************************
The accurate estimation of State of Charge (SoC), State of Health (SoH), and Remaining Useful Life (RUL) is critical for optimizing battery performance in automotive and railway applications. These metrics, defined as follows, underpin the reliability and efficiency of battery management systems (BMS):
\begin{itemize}
    \item \textbf{SoC}: The ratio of remaining charge to maximum capacity, expressed as \( \text{SoC} = \frac{\text{Remaining Charge}}{\text{Maximum Capacity}} \times 100\% \).
    \item \textbf{SoH}: The ratio of current maximum capacity to original capacity, given by \( \text{SoH} = \frac{\text{Current Maximum Capacity}}{\text{Original Maximum Capacity}} \times 100\% \).
    \item \textbf{RUL}: The number of cycles remaining before the battery's performance falls below a specified threshold, defined as \( \text{RUL} = \text{Total Expected Life} - \text{Current Age} \).
\end{itemize}
Traditional methods, such as Coulomb Counting and Kalman Filters, often struggle with nonlinearities and dynamic operating conditions. Recent advancements in Artificial Intelligence (AI), particularly machine learning and deep learning, offer promising solutions by capturing complex patterns in battery data. This chapter reviews the state of the art in SoC, SoH, and RUL estimation, focusing on AI-based approaches.

\textbf{Traditional Methods for SoC, SoH, and RUL Estimation}
Traditional methods for battery state estimation can be categorized into physics-based and statistical approaches, each with inherent limitations.

\textbf{Physics-Based Methods}
Physics-based methods model the electrochemical and electrical behavior of batteries. Key approaches include:
\begin{itemize}
    \item \textbf{Equivalent Circuit Models (ECMs)}: Represent batteries using electrical components (e.g., resistors, capacitors) to simulate voltage and current dynamics. ECMs are computationally efficient but lack precision under varying conditions \cite{Ref5}.
    \item \textbf{Electrochemical Models}: Simulate internal chemical reactions, offering high accuracy but requiring significant computational resources and detailed parameter knowledge \cite{Ref5}.
\end{itemize}

\textbf{Statistical Methods}
Statistical techniques rely on empirical data to estimate battery states. Common methods include:
\begin{itemize}
    \item \textbf{Coulomb Counting}: Integrates current over time to estimate SoC, expressed as \( \text{SoC}(t) = \text{SoC}(t_0) + \frac{1}{C_n} \int_{t_0}^t I(\tau) d\tau \), where \( C_n \) is nominal capacity. This method is sensitive to measurement errors and initial SoC inaccuracies \cite{Report}.
    \item \textbf{Kalman Filters}: Use recursive algorithms to refine state estimates by combining model predictions with noisy measurements. While effective for linear systems, they struggle with the nonlinear dynamics of batteries \cite{Report}.
\end{itemize}

These methods often fail to capture subtle variations in battery behavior under dynamic operating conditions, necessitating advanced approaches.

\textbf{AI-Based Methods for Battery State Estimation}
AI-based methods leverage machine learning and deep learning to model complex, nonlinear relationships in battery data. This section reviews key approaches, datasets, and their applications in automotive and railway contexts.

\textbf{Machine Learning Techniques}
Supervised machine learning algorithms, such as Support Vector Machines (SVMs) and Random Forests, have been applied to predict SoC and SoH using features like voltage, current, and temperature. For instance, \cite{Ref5} demonstrates SVMs achieving high accuracy in SoH estimation for lithium-ion batteries under controlled conditions. However, these methods require extensive feature engineering and struggle with temporal dependencies.

\textbf{Deep Learning Architectures}
Deep learning models excel at capturing temporal and spatial patterns in battery data. Key architectures include:
\begin{itemize}
    \item \textbf{Convolutional Neural Networks (CNNs)}: Extract spatial features from battery data, such as voltage profiles. Combining CNNs with Long Short-Term Memory (LSTM) units, as implemented in the \textit{BattAIHealth} project (see Section 4.1 of \cite{Report}), enhances forecasting accuracy by modeling temporal dependencies \cite{Ref4}.
    \item \textbf{Recurrent Neural Networks (RNNs) and LSTMs}: Designed for sequential data, LSTMs are particularly effective for RUL prediction, as they capture long-term degradation trends. Studies using the NASA Battery Dataset \cite{Ref1} report LSTM-based models outperforming traditional methods in RUL estimation.
    \item \textbf{Transformer Models}: Emerging in battery state estimation, transformers leverage attention mechanisms to model complex dependencies, showing promise in handling variable-length sequences \cite{Ref5}.
\end{itemize}

\textbf{Datasets for AI-Based Estimation}
The quality and diversity of datasets are critical for training robust AI models. Notable datasets include:
\begin{itemize}
    \item \textbf{NASA Battery Dataset} \cite{Ref1}: Provides voltage, current, temperature, and impedance data under various operating conditions, widely used for SoC and RUL estimation due to its diversity.
    \item \textbf{Aging Dataset from EV} \cite{Ref3}: Captures real-world electric vehicle profiles, including diagnostic tests like Hybrid Pulse Power Characterization (HPPC) and Electrochemical Impedance Spectroscopy (EIS), ideal for SoH and RUL studies.
\end{itemize}
These datasets highlight the importance of incorporating real-world operating conditions and diagnostic measurements to improve model generalizability.

\textbf{Hybrid Approaches}
Hybrid models combine physics-based and data-driven methods to enhance accuracy. For example, \cite{Ref5} integrates ECMs with neural networks to refine SoC estimates, leveraging physical constraints to reduce training data requirements. Such approaches are particularly relevant for railway applications, where operational conditions vary widely.


%************************************************
\chapter{Development}
\label{ch:Development}
%************************************************
\lipsum[1]
\section{Dataset Collection and Preprocessing}

o dataset principal utilizado neste trabalho é o dataset //calce, que foi optido por xxxx, da maneira xxxxx
falar da separacao do dataset
limpeza de dados
como é feito o input dos dados para o modelo
\section{Utilized Model}
o modelo utilizado é uma arquitetura open source de redes neuronais, especialistga em dados temporais, esta biliotecam, times net é uma das que tem mais accuracy em termos de estimar dados temporais
.... falar um pouco da arquitetura, e como foi utilizado, e o que foi feito da minha parte

\section{Model Opimization}
para a optimizacao do modelo foi utilizada a ferramenta optuna, que permite a optimizacao de hiperparametros de modelos de machine learning, e a sua integracao com o wandb, que permite a visualizacao dos resultados e comparacao entre os modelos
falar dos parametros que foram otimizados, e como foi feita a optimizacao, e o que foi aprendido com isso
falar dos parametros mais importantes




