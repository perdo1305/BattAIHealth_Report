
\subsection{monitoring battery health}
Battery health monitoring is critical for ensuring the reliability, safety, and longevity. Monitoring involves assessing the state of charge,
and the state of health (SOH),
como vimos no capitulo anterior....blablabla
% Technical Challenges section
\subsubsection{Technical Challenges}
The technical challenges in monitoring battery health arise from the complex nature of battery systems and the difficulties in accurately estimating SOC and SOH.

\subsubsection{Complexity of Battery Chemistry}
Batteries, particularly lithium-ion batteries, have intricate internal chemistries that are difficult to model and monitor.
Factors such as temperature, charge-discharge rates, and depth of discharge influence degradation, making accurate SOH estimation challenging. 
The nonlinear and complex degradation processes vary with usage conditions, environmental factors, and battery design, complicating predictive modeling.

\subsubsection{Measurement Difficulties}
Measuring individual battery parameters, such as internal resistance, temperature, and voltage, is technically challenging, especially in real-time applications. 
This requires precise sensors and sophisticated equipment, which may not be feasible in real-world scenarios. 
For instance, accurately measuring internal resistance or temperature in a moving vehicle is far more complex than in a controlled lab environment.

\subsubsection{Modeling and Estimation}
\textbf{rever isto !!!} \\
Developing accurate models for SOH estimation is complex. 
Electrochemical models, which simulate battery behavior based on physical and chemical principles, require extensive computational resources and detailed parameter inputs (e.g., electrolyte properties, reaction rates). 
Semi-empirical models often oversimplify electrochemical processes, reducing their effectiveness under extreme conditions. Equivalent circuit models (ECMs) may lack precision during high-rate charging/discharging or extreme temperatures due to their simplified nature.

\subsubsection{Limitations of Data-Driven Methods}
Data-driven approaches, such as machine learning techniques (e.g. Support Vector Regression, Gaussian Process Regression, Artificial Neural Networks), rely on large, high-quality datasets, which can be difficult to obtain. 
These methods also lack physical interpretability, making it difficult to understand their predictions. 
Additionally, issues like overfitting and high computational demands pose challenges for real-time applications.

\subsubsection{Complexity of Hybrid Methods}
Hybrid approaches, which combine model-based and data-driven methods, can improve accuracy but increase system complexity and computational costs. 
Interpreting errors in these systems remains a challenge, requiring further research to enhance transparency and efficiency.

% Practical Challenges section
\section{Practical Challenges}
Practical challenges stem from the gap between controlled laboratory environments and real-world operational conditions, as well as the need for scalable and reliable monitoring systems on edge devices.

\subsection{Laboratory vs. Real-World Conditions}
There is a significant discrepancy between laboratory-simulated conditions and actual operational environments. 
Laboratory settings often use sophisticated equipment that is not available in real-world applications, limiting the applicability of monitoring methods. For example, real-world conditions like varying temperatures or road vibrations are difficult to replicate in a lab, affecting SOH estimation accuracy.

\subsection{Scalability}
Monitoring systems must handle large numbers of batteries, especially in applications like EVs or trains. 
This requires scalable solutions that can process vast amounts of data efficiently. 
For instance, managing battery health across a fleet of electric vehicles demands robust, centralized data systems that can scale without compromising performance.

\subsection{Real-Time Monitoring}
Achieving real-time, reliable SOH monitoring is crucial for safety-critical applications but is technically demanding. 
Battery management systems (BMS) must balance accuracy with computational efficiency to provide timely insights without overloading system resources.

\subsection{Environmental Factors}
Batteries are sensitive to environmental conditions such as temperature, humidity, and vibration. 
Monitoring systems must account for these factors, which can significantly impact battery health and performance. 
For example, high temperatures can accelerate battery degradation, while low temperatures may reduce capacity, complicating health estimation.

% Economic and Feasibility Challenges section
\section{Economic and Feasibility Challenges}
Economic considerations and the feasibility of implementing advanced monitoring systems present additional hurdles.

\subsection{Cost of Monitoring Systems}
Implementing sophisticated battery health monitoring systems can be expensive, both in terms of initial setup and ongoing maintenance. 
This includes the cost of sensors, data storage, and computational infrastructure, which can be prohibitive for smaller organizations or applications.

\subsection{Data and Computational Costs}
AI and data-driven methods require significant computational resources and high-quality data, which can be costly to acquire and process. 
The high demand for data and computing power its a problem, particularly for real-time monitoring applications and edge devices.

\subsection{Standardization}
The lack of standardized methods and parameters for battery health monitoring hinders interoperability and comparison across different systems. 
This fragmentation makes it difficult to integrate monitoring solutions across diverse applications, such as consumer electronics and industrial systems.
